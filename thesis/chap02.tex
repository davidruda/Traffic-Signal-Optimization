\chapter{Simulator}

This chapter describes the simulator that we developed for the Traffic signaling problem introduced in Chapter~\ref{chap:problem_description}. Section~\ref{sec:overview} explains our motivation and the general design choices we made. Section~\ref{sec:functionality} describes the main features of the simulator package and shows examples of its usage. \xxx{For more details about the code and its implementation, please refer to the Developer Documentation in Appendix~\ref{chap:developer_documentation}.}

\section{Overview} \label{sec:overview}

To optimize the Traffic signaling problem, we implemented a custom simulator that calculates the score of a given configuration of traffic light schedules. The simulator is written in C++ to be as fast and efficient as possible, but is wrapped as a Python package using the powerful pybind11 library \cite{jakob2017pybind11}, making it very convenient to use.
Inspired by libraries such as NumPy\footnote{\url{https://numpy.org/}} and PyTorch\footnote{\url{https://pytorch.org/}}, our goal is to provide an easy-to-use interface and seamless integration with the vital Python ecosystem, without compromising on top-tier performance.

The Python package is called \verb|traffic-signaling|. It can be used as a standalone tool for an easy evaluation of solutions with features similar to the original competition scoring system, which is no longer available. Alternatively, it can be integrated into an optimization pipeline as a black-box scoring mechanism to further improve solutions, as we show in this thesis.

\section{Functionality} \label{sec:functionality}

There are two main classes in the \verb|traffic-signaling| package: \verb|CityPlan| and \verb|Simulation|. The \verb|CityPlan| class stores all static information of a given dataset from the input file. The \verb|Simulation| class uses the plan and provides the main functionality. It can:
\begin{itemize}
    \item Calculate the score and show the same statistics as the original competition scoring system.
    \item Import and export schedules in the competition file format.
    \item Create schedules using several different initialization options.
    \item Access and modify the non-trivial schedules in a format suitable for optimization.
\end{itemize} 

\begin{lstlisting}[language=Python]
from traffic_signaling import *

# Load the city plan for a specific dataset
plan = create_city_plan(data='e')
# Create a simulation for the given city plan
sim = Simulation(plan)
# Set traffic light schedules to default values
sim.default_schedules()
# Calculate the score
score = sim.score()
# Show statistics of the simulation
sim.summary()
\end{lstlisting}
