\chapter*{Introduction}
\addcontentsline{toc}{chapter}{Introduction}

%Zadání
%
%Simulace problému z Google Hash Code 2021 a jeho optimalizace pomocí různých technik umělé inteligence Genetický algoritmus, Local search, Simulated annealing a porovnání těchto přístupů
%
%Cílem je efektivně implementovat simulaci a vyzkoušet různé algoritmy pro její optimalizaci, nikoliv implementovat samotné optimalizační algoritmy (avšak pokud bude vhodnější genetické algoritmy/local search ručně naimplementovat než použít knihovnu, bude to možné, ale není to cílem)
%
%
% Zásady pro vypracování
% Student se ve své práci bude věnovat řízení množiny křižovatek zadané grafem dle specifikace Google Hash Code [1].
% Student práci začne implementací vlastního simulátoru křižovatek dle uvedené specifikace, který bude používat k optimalizaci řízení.
% K optimalizaci student s použitím vhodných nástrojů implementuje a následně experimentálně porovná následující algoritmy.
% 1) Evoluční algoritmy
% 2) Lokální prohledávání
% Na základě získaných zkušeností student dále zvolí některý z následujících přístupů.
% 3) Memetiské algoritmy
% 4) Metaheuristiky
% 5) Náhradní modely

Due to its significant impact on urban mobility and the environment, Traffic Signal Control (TSC) is a widely studied problem \cite{zhao2012computational}. As the number of vehicles on the road continues to increase \cite{caves2004encyclopedia}, its importance is growing even further.
Traffic signal optimization is known to be a cost-effective method for reducing congestion without physically changing the road infrastructure \cite{wang2021traffic}.
Optimized traffic lights reduce delays at intersections \cite{mu2022traffic}, which directly translates into time savings for commuters and improves the overall efficiency of transportation networks.
Additionally, TSC plays a critical role in reducing vehicle emissions \cite{gunarathne2023traffic}, helping to reduce pollution and promote environmental sustainability.

There are many approaches to solving TSC \cite{qadri2020state}. Early static, fixed-time designs have evolved into real-time adaptive systems and data-driven algorithms, utilizing data from various detectors and sensors. In practice, commonly used are adaptive traffic control systems (ATCS) such as SCOOT, SCATS, and SURTRAC \cite{smith2013surtrac}, which continuously adjust splits, offsets, and cycle lengths network-wide. In addition, optimization-based methods---such as Genetic Algorithm (GA) \cite{costa2020intersection}, Simulated Annealing (SA) \cite{qadri2020state}, and Model Predictive Control (MPC) \cite{ye2019survey}---have also been employed for traffic signal planning and coordination. More recently, learning-based methods, particularly those based on Reinforcement Learning (RL) and Deep Reinforcement Learning (DRL), are increasingly being explored as alternatives to traditionally used methods \cite{zhao2024survey, saadi2025survey}.

\section*{Contribution of the thesis}

In this thesis, we focus on the Traffic signaling problem from the Google Hash Code 2021 competition \cite{google2023google}. It serves as a simplified version of the real-world problem of traffic signal optimization in a city. First, we implement a fast and efficient C++ simulator of the problem and wrap it as a Python package to allow for easy use and integration with the wide ecosystem of Python libraries. Then, we use the simulator to optimize the traffic light schedules using three heuristic algorithms: Genetic Algorithm (GA), Hill Climbing (HC), and Simulated Annealing (SA). We experimentally compare the performance of these algorithms on provided datasets of different sizes and structures.

The thesis is structured as follows. Chapter~\ref{chap:problem_description} describes the Traffic signaling problem in detail, including the city layout, cars, traffic lights, and scoring system. Chapter~\ref{chap:optimization_methods} covers the theory of the optimization algorithms used in this thesis. Chapter ...

\xxx{Chapter~\ref{chap:experimental_results} presents the expertiments comparing the optimization algorithms on the provided datasets.}
\xxx{Appendix~\ref{chap:user_guide} provides a user guide detailing how to use the simulator, run optimization, and execute the scripts used to replicate our experiments.
Appendix~\ref{chap:developer_documentation} contains developer documentation describing the implementation details of the simulator and optimization.}

\xxx{Blackbox optimization}