\chapter*{Introduction}
\addcontentsline{toc}{chapter}{Introduction}

%Zadání
%
%Simulace problému z Google Hash Code 2021 a jeho optimalizace pomocí různých technik umělé inteligence Genetický algoritmus, Local search, Simulated annealing a porovnání těchto přístupů
%
%Cílem je efektivně implementovat simulaci a vyzkoušet různé algoritmy pro její optimalizaci, nikoliv implementovat samotné optimalizační algoritmy (avšak pokud bude vhodnější genetické algoritmy/local search ručně naimplementovat než použít knihovnu, bude to možné, ale není to cílem)
%
%
% Zásady pro vypracování
% Student se ve své práci bude věnovat řízení množiny křižovatek zadané grafem dle specifikace Google Hash Code [1].
% Student práci začne implementací vlastního simulátoru křižovatek dle uvedené specifikace, který bude používat k optimalizaci řízení.
% K optimalizaci student s použitím vhodných nástrojů implementuje a následně experimentálně porovná následující algoritmy.
% 1) Evoluční algoritmy
% 2) Lokální prohledávání
% Na základě získaných zkušeností student dále zvolí některý z následujících přístupů.
% 3) Memetiské algoritmy
% 4) Metaheuristiky
% 5) Náhradní modely

Traffic signal optimization is an important problem in today's world due to its
significant impact on urban mobility and the environment. Effective traffic
signal optimization reduces delays at intersections, which directly translates
to saving time for commuters and improving the overall efficiency of
transportation networks. By minimizing congestion, optimized traffic signals
facilitate smoother traffic flow, which not only enhances the daily commute
experience but also supports the economic vitality of urban areas by reducing
the time lost in traffic.

Additionally, efficient traffic signal timing plays a
critical role in reducing vehicle emissions, thereby contributing to lower
pollution levels and promoting environmental sustainability. Furthermore, as
urban populations continue to grow, the demand for more intelligent and
responsive traffic management systems increases, making traffic signal
optimization a key component in the development of smart cities and sustainable
urban infrastructures.

An~example citation: \cite{google2023google} \cite{rodrigues2023principled} \cite{li2022building} \cite{fortin2012deap}.

\section*{Related Work} \label{sec:related_work}
\addcontentsline{toc}{section}{Related Work}

\section*{Contribution of the Thesis} \label{sec:contribution_of_the_thesis}
\addcontentsline{toc}{section}{Contribution of the Thesis}
