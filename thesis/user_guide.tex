\chapter{User Guide} \label{chap:user_guide}

This guide is intended for users interested in running the code associated with this thesis. It is logically divided into two parts:
\begin{itemize}
    \item \textbf{Simulator} - a standalone tool for running the Traffic signaling problem (described in Chapter~\ref{chap:problem_description}) with various handy features, wrapped into the \verb|traffic_signaling| Python package 
    \item \textbf{Optimization} - scripts for running the optimization and experiments, built on top of the simulator
\end{itemize}

\section{Prerequisites}

TL;DR: If you just want to setup the environment for optimization, run
\begin{verbatim}
    make setup
\end{verbatim}

To build and install the \verb|traffic_signaling| package, you need:
\begin{itemize}
    \item \textbf{C++ compiler} with C++20 code support; e.g., gcc 11, clang 16, or MSVC 19.29 (or later versions)
    \item \textbf{Python} 3.10 or later
    \item \textbf{Python headers} (most likely already installed with Python)
    \begin{itemize}
        \item You can verify that the headers are available by inspecting their expected location with e.g.
\begin{verbatim}
    python3 -c "import sysconfig;
        print(sysconfig.get_path('include'))"
\end{verbatim}
        \item If not available, install the headers with e.g.
\begin{verbatim}
    sudo apt install python3-dev
\end{verbatim}
    \end{itemize}
\end{itemize}
For running the optimization, you also need to install packages from \\
the \verb|requirements.txt| file
% Assuming you are in the root directory of the repository, you can install the package with
% \begin{verbatim}
%     pip install ./traffic_signaling
% \end{verbatim}